\documentclass[a4paper]{article}
\usepackage[utf8]{inputenc}
\usepackage[natbib,sorting=none]{biblatex}
\usepackage{graphicx}
\usepackage{fancyhdr}
\usepackage{indentfirst}

\addbibresource{references.bib}
\newcommand{\comment}[1]{\textbf{Comment: #1}}

\begin{document}
\thispagestyle{empty}
\begin{figure}[!h]
	\centering
	\includegraphics[width=100mm]{Images/poli-logo.png}
\end{figure}
\hfill
\begin{center}
    \fontsize{18px}{6mm}\selectfont \textsc{\textbf{Software Engineering II project}}
\end{center}
\begin{center}
    \fontsize{12px}{4mm}\selectfont \textsc{Academic Year: 2018/2019}
\end{center}
\hfill
\hfill
\begin{figure}[!h]
	\centering
	\includegraphics[width=120mm]{Images/trackme-logo.png}
\end{figure}
\hfill
\hfill
\begin{center}
    \fontsize{22px}{8mm}\selectfont \textsc{\textbf{Requirement Analysis and\\ Specification Document}}
\end{center}
\begin{center}
    \fontsize{14px}{4mm}\selectfont \textsc{{Draft Version 0.5 - 9/11/2018 - Rev 3.0}}
\end{center}
\hfill
\hfill
\begin{center}
\fontsize{14px}{4mm}\selectfont \textsc{\textit{Davide Rutigliano -  903616}}
\end{center}

\begin{center}
\fontsize{14px}{4mm}\selectfont \textsc{\textit{Claudio Ferrante - 903417\\}}
\end{center}

\begin{center}
\fontsize{14px}{4mm}\selectfont \textsc{\textit{Davide Matta - 920349}}
\end{center}
\pagenumbering{roman}
\tableofcontents
\rfoot{\includegraphics[width=30mm]{Images/trackme-logo-mini.png}}

%%%%%%%%%%%%%%%%%%%%%%%%%%%%%%%%%%%%%%%%%%%%%%%%%%%%%%%%%%%%%%%%%%%%%%
\newpage
\pagestyle{fancy}
\pagenumbering{arabic}
\section{Scope and Purpose}
This document is intended for analyzing \textit{TrackMe} software application and specifications through analysis of test cases extracted from requirements explained in documents presented.

\section{Project Analized}
\paragraph{Authors:}
\begin{itemize}
    \item Lorenzo Francesco;
    \item Negri Virginia;
    \item Molteni Luca.
\end{itemize}
\paragraph{Repository:}
https://github.com/iPhra/LorenzoMolteniNegri \newline

To perform the acceptance on test cases we used all the documents provided by authors (i.e. RASD, DD and ITD). The latter has been primarily used for installation and setup instructions; first two instead have been used for extracting test cases in order to proceed with the analysis.

\newpage
\section{Installation Setup}
\subsection{Server Installation}
Backend installation was straightforward. However, database installation wasn't.

We had to ask further instructions to setup the database and explicitly ask to get a database dump: it was assumed the customer is able (and want) to recreate the database from scratch following the tables described in the document.

Moreover, it was not explained anywhere how to set server's database configuration for running tests.
\subsection{Client Installation}
Client installation was straightforward and quite simple in case you have a Mac.

\newpage
\section{Acceptance Test Case}
In this section, we will provide the analysis of the acceptance of the test cases we have extract from the documents provided. \newline\newline
\textbf{Registration}:\newline
The Registration functionalities work in a correct way. This feature has been tested properly, analysing the different cases, for what concerns general user registration: wrong password, email, FC, birth-date and sex. This has been done also for Third-Parties Registration, not allowing to register users who have submitted wrong IVA, company name and description.\newline
Beside has been verified that already existing users cannot do the registration again. Furthermore the application permits to activate the account properly.
\newline\newline
\textbf{Login}:\newline
The Login works correctly. Cases as, try to login as a not-existing user or to login providing a wrong password have been tested. Beside user whose account has not been activated cannot log into the system.
\newline\newline
\textbf{Profile Data Management}:\newline
The application allows users, both individuals and third-parties, to manage correctly their profile. In particular, they can retrieve and update their private settings. A limit we have found is that it is not possible to edit the submitted email.
\newline\newline
\textbf{Upload Data}:\newline
The system allows single users to upload new data correctly and to retrieve statistics about their imported data. Beside, it checks that wrong data  are not present (in terms of data type, timestamps, values).\newline
\textit{Note}: the requirements concerning the test case of visualizing data, in the RASD, can be unified in just one requirement, to avoid redundancy.\newline
 \newline\newline
\textbf{Make Requests}:\newline
The application allows Third-Parties to make in a correct way individual or group requests.\newline
In particular, for what concerns individual request, it forbid  to send a request to a not-existing user, or to an user with an other request pending, coming from the same Third-Party. Beside, the system does not validate requests with wrong subscription or duration value.\newline
The application checks that groups requests contains correct parameters.
\newline\newline
\textbf{Manage Requests}:\newline
The system allows users to accept/deny requests coming from Third-Parties, in a proper way. The case of wrong Request ID has been tested.
\newline\newline
\textbf{View Data of Accepted Requests}:\newline
The application allows Third-parties to correctly download the data of the users that have accept their requests. The system checks that the requests have been accepted and that the requests exist.
In the case of Group Requests, the system permits, in a correct way, to Third-Parties, to download their data. Beside, it checks that the parameters constraints match. 







\


\newpage
\section{Additional Points}
\begin{itemize}
    \item Standards compliance, found on RASD2 at page 73, does not seems to make sense: std. \textbf{D0-178C} is the primary document by which certification authorities such as FAA, EASA and Transport Canada approve \textbf{all commercial software-based aerospace systems}. We asked the development team an explanation, and they confirmed that it was indeed that standard: their system has been described as a \textbf{Level D risk level} which, according to online resources corresponds to \textit{"Failure slightly reduces the safety margin or slightly increases crew workload. Examples might include causing passenger inconvenience or a routine flight plan change."};
    
    \item The alloy model was poorly done. There are some error distributed all over the model which can be easily seen; for instance at \textit{Page 85 in Figure 35 of RASD2}: multiple users have the same FC and email, that is clearly wrong;
    
    \item iOS only was not the best choice for a prototype. That is because, if you don't have a macOS with Mojave and Xcode 10.1, you \textbf{can not test} the frontend. Furthermore, the Rasd says: \textit{"The implementation of the S2B as a mobile application ensures itself portability with regards to the user, since a porting from Android to iOS, and vice versa, would be quite natural."}: porting a native iOS application (Swift) to Android (Java) is not quite natural.
\end{itemize}

\newpage
\section{Effort Spent}
    \begin{itemize}
        \item[-] \textbf{Davide Rutigliano:}
        
        \item[-] \textbf{Davide Matta:}
        
        \item[-] \textbf{Claudio Ferrante:}
    \end{itemize}
\end{document}