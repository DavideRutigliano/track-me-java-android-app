\documentclass[a4paper]{article}
\usepackage[utf8]{inputenc}
\usepackage[natbib,sorting=none]{biblatex}
\usepackage{graphicx}
\usepackage{acronym}
\usepackage{indentfirst}
\usepackage{requirements} %custom package
\usepackage[none]{hyphenat} %stops word separation on new line
\usepackage{fancyhdr}
\usepackage{enumitem}
\usepackage{listings}
\usepackage{xcolor}

\definecolor{commentgreen}{RGB}{2,112,10}
\definecolor{eminence}{RGB}{108,48,130}
\definecolor{weborange}{RGB}{255,165,0}
\definecolor{frenchplum}{RGB}{129,20,83}

\addbibresource{references.bib}

\begin{document}
\setlist[enumerate]{label=\alph*),leftmargin=0.4cm, topsep=0cm}
\setlength{}{}
\pagenumbering{none}
\thispagestyle{empty}
\begin{figure}[!h]
	\centering
	\includegraphics[width=100mm]{Images/poli-logo.png}
\end{figure}
\hfill
\begin{center}
    \fontsize{18px}{6mm}\selectfont \textsc{\textbf{Software Engineering II project}}
\end{center}
\begin{center}
    \fontsize{12px}{4mm}\selectfont \textsc{Academic Year: 2018/2019}
\end{center}
\hfill
\hfill
\begin{figure}[!h]
	\centering
	\includegraphics[width=120mm]{Images/trackme-logo.png}
\end{figure}
\hfill
\hfill
\begin{center}
    \fontsize{22px}{8mm}\selectfont \textsc{\textbf{Requirement Analysis and\\ Specification Document}}
\end{center}
\begin{center}
    \fontsize{14px}{4mm}\selectfont \textsc{{Draft Version 0.5 - 9/11/2018 - Rev 3.0}}
\end{center}
\hfill
\hfill
\begin{center}
\fontsize{14px}{4mm}\selectfont \textsc{\textit{Davide Rutigliano -  903616}}
\end{center}

\begin{center}
\fontsize{14px}{4mm}\selectfont \textsc{\textit{Claudio Ferrante - 903417\\}}
\end{center}

\begin{center}
\fontsize{14px}{4mm}\selectfont \textsc{\textit{Davide Matta - 920349}}
\end{center}
\pagenumbering{roman}
\tableofcontents
\newpage
\addcontentsline{toc}{subsection}{Use Cases Summary}
\renewcommand\listtablename{Use Cases Summary}
\listoftables
\newpage
\addcontentsline{toc}{subsection}{UML Diagrams List}
\renewcommand\listfigurename{UML Diagrams List}
\listoffigures
\newpage
\rfoot{\includegraphics[width=30mm]{images/trackme-logo-mini.png}}
%%%%%%%%%%%%%%%%%%%%%%%%%%%%%%%%%%%%%%%%%%%%%%%%%%%%%%%%%%%%%%
\newpage
\pagestyle{fancy}
\pagenumbering{arabic}
\section{Introduction}

    \subsection{Purpose}
    This Requirement Analysis and Specification Document addresses the description in terms of functional and non-functional requirements of \textit{TrackMe}, a new health monitoring application.
    
    This document is aimed at giving an explanation both of the problem and of the proposed software-based solution by pointing out different features of the application. Moreover, this specifications also addresses some relevant QoS characteristics that the system should guarantee.
    
    Thus, the analysis is primarily focused on requirements elicitation and use cases description through well known standards for specification documents:
        \begin{itemize}
            \item[\textit{1)}] \textit{UML} for Use Case, Sequence and Activity Diagrams \cite{rumbaugh2004unified};
            \item[\textit{2)}] \textit{Alloy} for formal analysis \cite{jackson2006software}.
        \end{itemize}
        
    This document is mostly directed and likely necessary to software designers and developers interested in the deployment of the proposed system.
    
    \subsection{Scope and Audience}
        \subsubsection{Description of the given problem}
        
        \subsubsection{Description of the proposed system}
        \textit{TrackMe} is an ease of use health monitoring application which offers different services for both young and old people who needs to keep track of their personal data in order to to keep their health safe.
        
        This proposed system provides a very efficient data acquisition service named \textit{Data4Help}. This facility gives the possibility to verified \textit{third-party} registered onto the service to request health status information about subscribed customers. The application may also allow \textit{individual} users to connect \textit{external devices} such as smart-watches or similar, to perform a more detailed data acquisition and monitoring.
        
        In addition, the application provides also a fully automated SOS service for elderly people: \textit{Automated-SOS} uses \textit{Data4Help} features to monitor the health status and, when such parameters are below certain thresholds, sends the location of the customer to an ambulance.
    
        Moreover, through the \textit{Track4Run} service, \textit{TrackMe} allows customers to create a new run, a competition in which other users can either participate as \textit{athletes} or watch as \textit{spectators}. Additionally, this service may exploits other service functionality for keeping track of athletes progresses and for helping them to prevent major accident happen during the run. 
        \subsubsection{Goals}
        
        \begin{description}
            \item[G.01] Goal
        \end{description}
    
    \subsection{Document Overview}
        This part of the document is intended to provide both an overview of the problem and an idea of the proposed solution.
        
        Next section is aimed at giving a more detailed description of the proposed system, by meaning of the application point of view. Additionally, addresses users characteristic, dependencies and constraint.
        
        Third chapter shows a detailed analysis of \textit{TrackMe} in terms both functional and non-functional requirements of the system. Furthermore, takes into account definition of all use cases, their description through UML diagrams and their mapping on requirements.
        
        Fourth section illustrates \textit{Alloy} models including purpose, proof and explanation. In addition, describes worlds obtained by running them.
        
        Last part is about effort spent in writing this document in terms of hours of work.
        
        There is also a list of tables and a list of figures representing respectively use cases and diagrams in order to help the reader to understand them and navigate the specification.
        
    \subsection{Definition, Acronyms and  Abbreviations}
            \subsubsection{Keywords}
            The key words “MUST”, “MUST NOT”, “REQUIRED”, “SHALL”, “SHALL NOT”, “SHOULD”, “SHOULD NOT”, “RECOMMENDED”, “MAY”, and “OPTIONAL” in this document are to be interpreted as described in RFC 2119 \cite{bradner1997key}.
            \subsubsection{Definition of Terms}
            This document uses several terms that might be more loosely used elsewhere. These terms are defined here as they will be used later on in this document.
                \begin{description}
                    \item[\textbf{User}] a general definition of a customer registered into the application
                    
                    \item[\textbf{Individual}] a single person \textit{TrackMe User}, identified either by SSN or by FC number
                    
                    \item[\textbf{Third-Party}] a company \textit{TrackMe User}, identified by VAT number
                    
                    \item[\textbf{External Device}] an external device such as a smart-watch or similar devices
                    
                    \item[\textbf{Run}] a competition organized and managed by an \textit{Organizer} to which \textit{Athletes} and \textit{Spectators} can enroll as participant or watchers
                    
                    \item[\textbf{Organizer}] either an \textit{Individual} or a \textit{Third-Party} that uses \textit{Track4Run} service to create a new \textit{Run}
                    
                    \item[\textbf{Athlete}] an \textit{Individual} enrolled to an already existing \textit{Run}
                    
                    \item[\textbf{Spectator}] either an \textit{Individual} or a \textit{Third-Party} that is watching an already existing \textit{Run}
                    
                    \item[\textbf{TrackMe}] the \textit{"system to be"}
                    
                    \item[\textbf{Data4Help}] a data monitoring service provided by \textit{TrackMe}
                    
                    \item[\textbf{Automated-SOS}] an SOS service built on top of \textit{Data4Help}
                    
                    \item[\textbf{Track4Run}] run management service offered by \textit{TrackMe} application
                    
                    \item[\textbf{Credential}] as used in this document, is a combination of both username and password used by an \textit{User} to authenticate him/herself during the Log-in phase
                \end{description}
                
            \subsubsection{Abbreviations}
            \begin{itemize}
                \item[-] Abbreviation 
            \end{itemize}
            
            \subsubsection{Acronyms}
            \begin{acronym}
                \acro{RASD}{Requirement Analysis and Specification Document}
                \acro{UML}{Unified Modelling Language}
                \acro{QoS}{Quality of Service}
                \acro{SSN}{Social Security Number}
                \acro{FC}{Fiscal Code}
                \acro{VAT}{Value Added Tax}
                \acro{BT}{Bluetooth}
                \acro{NFC}{Near Field Communication}
            \end{acronym}
            
    \subsection{References}
        \printbibliography[heading=none]
%%%%%%%%%%%%%%%%%%%%%%%%%%%%%%%%%%%%%%%%%%%%%%%%%%%%%%%%%%%%%%
\newpage
\section{Overall Description}

    \subsection{Product Perspective}
    
    \subsection{Product Functions}
    
        \subsubsection{Data4Help}
        
        \subsubsection{Automated-SOS}
        
        \subsubsection{Track4Run}
    
    \subsection{User Characteristics}
    
        \subsubsection{Individual}
        
        \subsubsection{Third-Party}
        
    \subsection{Assumptions, Dependencies and Constraints}
        \begin{description}
            \item[A.01] Assumption
        \end{description}
        
        \begin{description}
            \item[D.01] Dependencies
        \end{description}
        
        \begin{description}
            \item[C.01] Constraints
        \end{description}
%%%%%%%%%%%%%%%%%%%%%%%%%%%%%%%%%%%%%%%%%%%%%%%%%%%%%%%%%%%%%%
\newpage
\section{Specific Requirements}

    \subsection{External Interface Requirements}
        
        \subsubsection{User Interfaces}
        
        \subsubsection{Hardware Interfaces}
        
        \subsubsection{Software Interfaces}
        
        \subsubsection{Communication Interfaces}
    
    \subsection{Functional Requirements}
        
        \subsubsection{Registration}
        \textit{TrackMe} must provide a registration interface for both \textit{individual} and \textit{third-party} users.
        
        An \textit{individual} must provide his/her SSN (Social Security Number) or FC (Fiscal Code) during the registration phase in order to be uniquely identified by \textit{TrackMe} application. In this way, the user is also accepting terms of service and allows \textit{TrackMe} to acquire his/her personal data.
        %use cases table structure
        \begin{usecase}{Individual User Registration}
        \name{Individual User Registration}
        \brief{An individual sign up to the application to become an user.}
        \actor{Individual, TrackMe.}
        \pre{The individual has a valid SSN/FC and he's on the main page.}
        \bflow{The individual goes on registration page pressing \textit{'Sign-up'} button.}
              {TrackMe asks to the user if he's an Individual or a Third Party. The user can choose between these two options.}
              {The individual choose the Individual option.}
              {TrackMe asks to provide a SSN/FC (based on region) and user personal information such as name, surname, age, gender, e-mail, nationality and optional notes about his health condition.}
              {The individual fills out the requested information and presses the confirm button.}
              {TrackMe registers that user as an Individual User and adds it to the  Database.}
        \post{The individual is successfully registered as an Individual User.}
        \end{usecase}
        
        A \textit{third-party} must provide a VAT number during the registration in order to allow \textit{TrackMe} to check that the submitted VAT is registered and that is issued by an EU Member State.
        
        \begin{usecase}{Third Party User Registration}
        \name{Third Party User Registration}
        \brief{A third party sign up to the application to become an user.}
        \actor{Third Party, TrackMe.}
        \pre{The third party has a valid VAT and it's on the main page.}
        \bflow{The third party goes on registration page pressing \textit{'Sign-up'} button.}
              {TrackMe asks to the user if he's an Individual or a Third Party. The user can choose between these two options.}
              {The third party choose the Third Party option.}
              {TrackMe asks if user is an Individual or a Third Party. The user can choose between these two options.}
              {TrackMe asks to provide a VAT number and organization personal information.}
              {TrackMe registers the user as a Third Party user and adds it to the Database.}
        \post{The third party is successfully registered as an Third Party User.}
        \end{usecase}
        
        \subsubsection{Log-in}
        The system should provide a login interface to the \textit{User} in order sign into the application.
        
        \begin{usecase}{Individual User Login}
        \name{Individual User Login}
        \brief{An individual sign in to access the application.}
        \actor{Individual, TrackMe.}
        \pre{The individual is already registered and it's on the main page.}
        \bflow{The individual goes on login page pressing \textit{'Sign-in'} button.}
              {TrackMe provides to the \textit{user} a form composed by two text boxes where the user can write and a button to send the form. The first box is the \textit{'username'}, while latter is the \textit{'password'} field.}
              {The individual writes his username on the \textit{'username'} box and his associated password on \textit{'password'} box and presses the Log-in button.}
              {TrackMe checks username is on the DB and that its associated password is correct.}
              {TrackMe finds that username and its associated password are valid and grants access to the user.}
        \post{The individual is successfully authenticated as \textit{Individual User}.}
        \end{usecase}
        
        \begin{usecase}{Third Party User Login}
        \name{Third Party User Login}
        \brief{A third party sign in to access the application.}
        \actor{Third Party, TrackMe.}
        \pre{The third-party is already registered and it's on the main page.}
        \bflow{The third party goes on login page pressing \textit{'Sign-in'} button.}
              {TrackMe provides to the \textit{user} a form composed by two text boxes where the user can write and a button to send the form. The first box is the \textit{'username'}, while latter is the \textit{'password'} field.}
              {The third-party writes his username on the \textit{'username'} box and his associated password on \textit{'password'} box and presses the Log-in button.}
              {TrackMe checks username is on the DB and that its associated password is correct.}
              {TrackMe finds that username and its associated password are valid and grants access to the user.}
        \post{The third party is successfully authenticated as \textit{Third-Party User}.}
        \end{usecase}
        
        \subsubsection{Data4Help Request}
        The application may provide an interface that allows \textit{Third-Parties} to send requests to view the data of a specific \textit{individual} or a group of \textit{individuals}. A single individual may accept or refuse this request.
        
        \begin{usecase}{Individual Data Request}
        \name{Third Party Individual Data Request}
        \brief{A third party sends a request to an individual to view his data.}
        \actor{Third Party, Individual, TrackMe.}
        \pre{The third-party is logged-in, it's on the main page and has a valid SSN/FC of a registered individual.}
        \bflow{The third party presses the \textit{'Data-Request'} button.}
              {TrackMe asks if user wants to request data to a specific individual or to get anonymous data of a group of individuals.}
              {The third party presses the button \textit{'Individual'} }
              {TrackMe provides a box called \textit{'SSN/FC'} where the user can write.}
              {The third party writes the individual \textit{'SSN/FC'} and presses the \textit{'Submit'} button.}
              {TrackMe sends the data request to the individual.}
              {TrackMe informs the Third-Party user that the operation succeded.}
        \post{The third party has successfully sent a data request to the individual.}
        \end{usecase}
        
        \begin{usecase}{Third Party Group Data View}
        \name{Third Party Group Data View}
        \brief{A third party views data of a group of individuals.}
        \actor{Third Party, Individual, TrackMe.}
        \pre{The third-party is logged-in and it's on the main page.}
        \bflow{The third party presses the \textit{'Data-Request'} button. Third party is now on the Data-Request            page.}
              {TrackMe asks if user wants to request data to a specific individual or to get anonymous data of a group of individuals.}
              {The third party presses the button \textit{'Group'}. Third party is now on the Group subsection of Data-Request page.}
              {TrackMe provides a set of boxed called \textit{'Minimum Age'}, \textit{'Maximum Age'}, \textit{'Included Countries'}, \textit{'Excluded Countries'}.}
              {.....}
              {Third party presses the button called \textit('Confirm').}
              {TrackMe checks if the magnitude of the group defined on point \textit{e)} is greater than 1000.}
              {TrackMe informs the Third-Party user that the operation succeded.}
        \post{The third party has access to the anonymized data of a specific group of individuals.}
        \end{usecase}
        
        \begin{usecase}{Third Party Subscribes to New Data}
        \name{Third Party Subscribes to New Data}
        \brief{A third party subscribes to the data of a certain group of individuals.}
        \actor{Third Party, Individual, TrackMe.}
        \pre{The third-party is logged-in and is on the Group subsection of Data-Request page.}
        \bflow{TrackMe provides a set of boxed called \textit{'Minimum Age'}, \textit{'Maximum Age'}, \textit{'Included       Countries'}, \textit{'Excluded Countries'}.}
              {.....}
              {Third party presses the button called \textit('Subscribe'). Then, it presses the button called \textit('Confirm').}
              {TrackMe checks if the magnitude of the group defined on point \textit{e)} is greater than 1000.}
              {TrackMe informs the Third-Party user that the operation succeded.}
        \post{The third party is subscribed to the group data and will receive new data as soon as they are produced.}
        \end{usecase}
        
        \begin{usecase}{Individual Accepts Data Request}
        \name{Individual Accepts Data Request}
        \brief{An individual accepts the request to view his personal data.}
        \actor{Third Party, Individual, TrackMe.}
        \pre{The individual is logged-in and it's on the main page.}
        \bflow{TrackMe sends to the IU a request to view his data. The IU can see that the request was sent by a              specific TPU. TrackMe provides two buttons called \textit{Accept} and \textit{Refuse}.}
              {The IU presses the accept button.}
              {TrackMe sends to the TPU a confirmation about his request for that Individual.}
              {TrackMe informs the IU that the operation succeded.}
        \post{The third party has access to the Individual data.}
        \end{usecase}
        
        \subsubsection{Connection of External Devices}
        The application may provide an interface that allows \textit{individual} to connect an external device such as smart-watch via Bluetooth or NFC.
        
        Moreover, in case such interface is present, \textit{TrackMe} should be able to interact also with the external device and to acquire data from it.
        
        
        \subsubsection{Automated-SOS enabling}
        
        \begin{usecase}{third party enables Automated-SOS}
        \name{Third-Party enables Automated-SOS}
        \brief{A Third-Party wants to enable Automated-SOS, a service that guarantees that an ambulance will be called  in 5 seconds if the individual's health parameters are below certain thresholds.}
        \actor{Third party, \textit{TrackMe}}
        \pre{The Third-Paty is logged-in, it's on the main page and has not enabled Automated-SOS.}
        \bflow{The Third-Party presses the \textit{'enable Automated-SOS'} button.}
              {TrackMe checks if the individual has already enabled the service.}
           
        \post{Third-Party has enabled Automated-SOS and all users subscribed to the Third-Party are also subscribed to Automated-SOS}
        \end{usecase}
        \subsubsection{Automated-SOS Data Management}

        \begin{usecase}{Automated-SOS Data Management}
        \name{Automated-SOS Data Management}
        \brief{The system will periodically check indiviadual's data, and if the thresholds have been overcomed, it will automatically notify an ambulance within 5 seconds }
        \actor{\textit{TrackMe, ambulance} }
        \pre{At least one ambulance must be available; the Third.Party must have enabled Automated-SOS. The individual have to be subscribed to the Third-Party}
        \bflow{The system notice that an individual has overcame the thresholds.}
              {Track me checks in the ambulance database the nearest one available .}
              {The system sends to the selected ambulance a request with the GPS position of the subject in danger }
              {The ambulance selected should accept the request, otherwise TrackMe will notify the second ambulance available and so on until the request is accepted .}
        \post{An ambulance will go to the GPS position.}
        \end{usecase}
        

        
        \subsubsection{Run Creation}
         \begin{usecase}{A Third-Party creates a run}
        \name{Run Creation}
        \brief{A Third-Party creates a run using \textit{TrackMe} }
        \actor{Third party, \textit{TrackMe}}
        \pre{The Third-Paty is logged-in, it's on the \textit{Track4Run} section}
        \bflow{The Third-Party presses the \textit{'create Run'} button.}
              {The Third-Party insert name, date,time of the run.}
    {The Third-Party, using an interactive map, create the path of the run}
              {The Third party clicks on the \textit{'Submit'} button }
              {TrackMe updates the database of the runs}
           
        \post{The run has been created and now the subscribers of the Third-Party can see it on TrackMe}
        \end{usecase}
        \subsubsection{Run Enrollment}
        \begin{usecase}{An user enrolls to a run}
        \name{Run Enrolling}
        \brief{A single user subscribed to a Third-Party can enroll to a run created by it, using the Track4Run section of TrackMe }
        \actor{Individual, \textit{TrackMe}}
        \pre{The Individual is logged-in, it's on the \textit{Track4Run} section, there is at least a run available}
        \bflow{The user presses the \textit{'enroll Run'} button.}
              {Track me select all the available runs, taken from the one created by the Third-Party which the user is subscribed.}
              {The individual selects the run he wants to enroll}
              {The user press the {'enroll'} button }
              {TrackMe updates the the number of subscribers for the selected run}
           
        \post{The individual is registered in the run and the number of partecipants is updated}
        \end{usecase}
        \subsubsection{Run Watching }
        \begin{usecase}{An user select a run to watch}
        \name{Run Watching selection}
        \brief{A single user subscribed to a Third-Party can select a run to watch created by it, using the Track4Run section of TrackMe }
        \actor{Individual, \textit{TrackMe}}
        \pre{The Individual is logged-in, it's on the \textit{Track4Run} section, there is at least a run available}
        \bflow{The user presses the \textit{'select run to watch'} button.}
              {TrackMe select all the available runs, taken from the ones created by the Third-Party which the user is subscribed, and prints out the output to the user.}
              {The individual selects the run he wants to watch}
              {The user press the {'select'} button }
              {TrackMe add the individual to a list of watchers and increases by one the number of observers of the run}
           
        \post{The individual can now watch the run he selected}
        \end{usecase}
        
        
        
        
        
        
        \begin{usecase}{An user can see the runners on the map}
        \name{Run Watching }
        \brief{An user can see the runners on the path of the run using the tools of TrackMe }
        \actor{Individual, \textit{TrackMe}}
        \pre{The Individual is logged-in, it's on the \textit{Track4Run} section and he has already selected at least one run; the runners should wear track devices}
        \bflow{The user presses the \textit{'watch Run'} button.}
              {TrackMe prints out the ongoing runs already selected by the watcher in the other section .}
              {The individual selects the run he wants to watch}
              {The user press the {'select'} button }
              {TrackMe provide to the user the names of the runners}
              {The individual selects a runner and press the \textit{'Track'} button}
              {Track me, shows on an interactive map the position of the runner on the path of the run, using the GPS position provide by the devices worn by the runners}
           
        \post{The individual can now se the position of the runners in real-time}
        \end{usecase}
        
    
    \subsection{Design Constraints}
    
        \subsubsection{Standards Compliance}
        
        \subsubsection{Hardware Limitations}
        
        \subsubsection{Other Constraints}
    
    \subsection{Software System Attributes}
        
        \subsubsection{Reliability}
        
        \subsubsection{Availability}
        
        \subsubsection{Security}
        
        \subsubsection{Maintainability}
        
        \subsubsection{Portability}
        
%%%%%%%%%%%%%%%%%%%%%%%%%%%%%%%%%%%%%%%%%%%%%%%%%%%%%%%%%%%%%%
\newpage
\section{Formal Analysis using Alloy}
\renewcommand*{\lstlistlistingname}{Alloy model samples}
\lstlistoflistings
\newpage
%definition of alloy language for Latex
\lstdefinelanguage{Alloy}{
	morekeywords={
		module, open, as,
		private, abstract, sig, extends, in,
		lone, some, one, disj,
		fact, pred, fun, assert,
		run, check,
		for, but, exactly,
		this, not, implies, else, let,
		not, no, set, all, sum,
		iff, or, Int, and,
		none, univ, iden,
	},
	sensitive=true,
	morecomment=[l]{//},
	morecomment=[l]{--},
	morecomment=[s]{/*}{*/},
	morestring=[b]{"},
	literate={->}{$\rightarrow$}1
	% replacing characters can cause problems when copying from PDF to editor
}[keywords,comments,strings]
\lstset {
    language=Alloy,
    frame=tb,
    tabsize=3,
    showstringspaces=false,
    breaklines=true,
    numbers=none,
    commentstyle=\color{commentgreen},
    keywordstyle=\color{blue},
    stringstyle=\color{red},
    basicstyle=\small\ttfamily, % basic font setting
    escapechar=\&,
    % keyword highlighting
    classoffset=1, % starting new class
    keywordstyle=\color{weborange},
    classoffset=0,
}
\subsection{Data4Help}
    \lstinputlisting[language=Alloy, caption=Data4Help: model definition, firstline=0, lastline=79]{TrackMeAlloyModel.als}
    
    \lstinputlisting[language=Alloy, caption=Data4Help: single individual request, firstline=80, lastline=95]{TrackMeAlloyModel.als}
    
    \begin{figure}[!htpb]
    	\centering
    	\includegraphics[width=100mm]{images/alloy/makeOneRequest.png}
    \end{figure}
    
    \lstinputlisting[language=Alloy, caption=Data4Help: complete, firstline=96, lastline=110]{TrackMeAlloyModel.als}
    
    \begin{figure}[!htpb]
    	\centering
    	\includegraphics[width=100mm]{images/alloy/Data4HelpComplete.png}
    \end{figure}

\subsection{Automated-SOS}
    \lstinputlisting[language=Alloy, caption=Automated-SOS: model definition, firstline=112, lastline=123]{TrackMeAlloyModel.als}
    
    \lstinputlisting[language=Alloy, caption=Automated-SOS: enable automated-sos, firstline=124, lastline=133]{TrackMeAlloyModel.als}
    
    \begin{figure}[!htpb]
    	\centering
    	\includegraphics[width=100mm]{images/alloy/enableAutomatedSos.png}
    \end{figure}
    
    \lstinputlisting[language=Alloy, caption=Automated-SOS: run Automated-SOS, firstline=134, lastline=145]{TrackMeAlloyModel.als}
    
    \begin{figure}[!h]
    	\centering
    	\includegraphics[width=100mm]{images/alloy/runAutomatedSos.png}
    \end{figure}
    \newpage
    \lstinputlisting[language=Alloy, caption=Automated-SOS: complete, firstline=146, lastline=158]{TrackMeAlloyModel.als}
    
    \begin{figure}[!htpb]
    	\centering
    	\includegraphics[width=100mm]{images/alloy/AutomatedSos+Data4HelpComplete.png}
    \end{figure}

\subsection{Track4Run}
\lstinputlisting[language=Alloy, caption=Track4Run: model definition, firstline=160, lastline=205]{TrackMeAlloyModel.als}

\lstinputlisting[language=Alloy, caption=Track4Run: create a new run, firstline=206, lastline=214]{TrackMeAlloyModel.als}

\begin{figure}[!htpb]
	\centering
	\includegraphics[width=100mm]{images/alloy/createNewRun.png}
\end{figure}

\lstinputlisting[language=Alloy, caption=Track4Run: enroll to a run, firstline=215, lastline=223]{TrackMeAlloyModel.als}

\begin{figure}[!htpb]
	\centering
	\includegraphics[width=100mm]{images/alloy/enrollToRun.png}
\end{figure}

\lstinputlisting[language=Alloy, caption=Track4Run: watch run, firstline=224, lastline=236]{TrackMeAlloyModel.als}

\begin{figure}[!htpb]
	\centering
	\includegraphics[width=100mm]{images/alloy/watchRun.png}
\end{figure}

\lstinputlisting[language=Alloy, caption=Track4Run: complete, firstline=237, lastline=247]{TrackMeAlloyModel.als}
\newpage
\begin{figure}[!htpb]
	\centering
	\includegraphics[width=100mm]{images/alloy/Track4RunComplete.png}
\end{figure}

\lstinputlisting[language=Alloy, caption=TrackMe: complete, firstline=248, lastline=261]{TrackMeAlloyModel.als}
\newpage
\begin{figure}[!htpb]
	\centering
	\includegraphics[width=200mm, angle = 90]{images/alloy/TrackMeComplete.png}
\end{figure}
%%%%%%%%%%%%%%%%%%%%%%%%%%%%%%%%%%%%%%%%%%%%%%%%%%%%%%%%%%%%%%
\newpage
\section{Effort Spent}
    \begin{itemize}
        \item[-] \textbf{Davide Rutigliano:}
        
        \item[-] \textbf{Davide Matta:}
        
        \item[-] \textbf{Claudio Ferrante:}
    \end{itemize}

\end{document}
